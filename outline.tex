\documentclass[twocolumn]{article}
\PassOptionsToPackage{english}{babel}
\usepackage{standard}
\usepackage{times}
\usepackage{hyperref}
\geometry{a4paper,left=20mm,right=20mm,bottom=30mm,top=30mm}
\pagestyle{empty}
\urlstyle{same}

\begin{document}
  \twocolumn[%
    \begin{center}
      \large
      \textbf{Presentation: Simulating the Evolution of Binary Black Holes}\\
      \textbf{Outline}
    \end{center}
    \medskip
    \tableofcontents
  ]
  \cleardoublepage

  \section{Introduction} % (fold)
  \label{sec:introduction}
    \begin{itemize}
      \item show video of simulation
      \item What is numerical relativity?
      \item Why do we need it?
    \end{itemize}
  % section introduction (end)

  \section{Background} % (fold)
  \label{sec:background}
    \begin{itemize}
      \item spacetime, manifolds, metrics
      \item tensors, curvature, geodesics
      \item Einstein field equations
      \item analytical solutions
        \begin{itemize}
          \item Schwarzschild black holes
          \item Kerr black holes
        \end{itemize}
    \end{itemize}
  % section background (end)

  \section{Basic Idea} % (fold)
  \label{sec:basic_idea}
    \begin{itemize}
      \item What do we have to do for the simulation?
      \item Why do we have to do this?
    \end{itemize}
    \begin{itemize}
      \item 3+1 decomposition
      \item construction of initial data
      \item choosing coordinates
      \item matter sources
      \item locating black hole horizons
      \item recasting the evolution equations
      \item numerical methods
      \item binary black hole evolution
    \end{itemize}
  % section basic_idea (end)

  \section{$3+1$ Decomposition of Einstein's Equations} % (fold)
    Use the directly the notion of basis vectors.

    \begin{itemize}
      \item Why do we have to decompose Einstein's equations?
      \item foliations of spacetime
      \item extrinsic curvature
      \item equations of Gauss, Codazzi and Ricci
      \item constraint and evolution equations
      \item ADM equations
    \end{itemize}

    What?

    Why?
    \begin{itemize}
      \item Einstein equations are complicated and do not want to be solved: contain time derivatives of second order, cannot be categorized to find solution method
      \item simulating 4 dimensions will need crazy amount of computation power and storage because dimension appears in exponent of grid point count
      \item separating time will result in time evolution (without saving many time slices)
      \item spatial equations can be categorized with better conditions
    \end{itemize}

    \subsection{Foliation of Spacetime} % (fold)
    \label{sub:foliation_of_spacetime}
      \begin{itemize}
        \item carving $(M,g_{ab})$ into stack of spatial (spacelike) $3$-dimensional slices $Σ$
        \item construct spatial metric $γ_{ab}$ and time normal $\Omega_a$ (together with lapse function $α$)
        \item find method to project spacetime tensors to their spatial and timelike part
        \item define spatial covariant derivative
        \item construct spatial Riemann tensor (describes intrinsic curvature of spatial slice)
        \item describe extrinsic curvature $K_{ab}$
        \item conditions for $γ_{ab}$ and $K_{ab}$: equations of Gauss, Codazzi and Ricci
      \end{itemize}
    % subsection foliation_of_spacetime (end)

    \subsection{The Constraint and Evolution Equations} % (fold)
    \label{sub:the_constraint_and_evolution_equations}
      \begin{itemize}
        \item use Einstein field equations together with Gauss and Codazzi equation
        \item derive Hamiltonian and momentum constraint
        \item define shift vector $β^a$ to create natural time derivative
        \item use equations of Einstein, Gauss and Ricci to derive evolution equation of extrinsic curvature
        \item derive evolution equation for spatial metric
      \end{itemize}
    % subsection the_constraint_and_evolution_equations (end)

    \subsection{Choosing Basis Vectors} % (fold)
    \label{sub:choosing_basis_vectors}
      \begin{itemize}
        \item formulate ADM equations based on $α$ and $β^i$
        \item examples: Schwarzschild and Kerr black holes
        \item two comments
      \end{itemize}
    % subsection choosing_basis_vectors (end)
  % section  (end)

  \section{Constructing initial data} % (fold)
  \label{sec:constructing_initial_data}
    \begin{itemize}
      \item conformal transformations
      \item example methods: conformal transverse-traceless decomposition, conformal thin-sandwich decomposition
      \item conformal thin-sandwich decomposition
      \item realistic data modelling
    \end{itemize}

    Why?
    \begin{itemize}
      \item physical realistic conditions for binary black holes
    \end{itemize}

    \begin{itemize}
      \item initial values to simulate time evolution from a starting point
      \item initial values have to fulfill constraint equations
      \item $12-4 = 8$ degrees of freedom, $4$ degrees for spacetime translations, leaving $4$ degrees of freedom for characterizing gravitational field
      \item separation of longitudinal and transversal part not possible
    \end{itemize}

    \subsection{Conformal transformations} % (fold)
    \label{sub:conformal_transformations}
      \begin{itemize}
        \item construct conformally related metric $\bar{γ}_{ij}$ based on conformal factor $ψ$
      \end{itemize}
    % subsection conformal_transformations (end)
  % section constructing_initial_data (end)

  \section{Choosing coordinates} % (fold)
  \label{sec:choosing_coordinates}
    \begin{itemize}
      \item gauge variables: lapse and shift
      \item geodesic slicing
      \item example methods: maximal slicing and singularity avoidance, harmonic coordinates
    \end{itemize}
  % section choosing_coordinates (end)

  \section{Recasting the Evolution Equations} % (fold)
  \label{sec:recasting_the_evolution_equations}
    \begin{itemize}
      \item instability problem of ADM evolution equations
      \item example methods: generalized harmonic coordinates, first-order symmetric hyperbolic formulations, BSSN formulation
      \item BSSN formulation
    \end{itemize}
  % section recasting_the_evolution_equations (end)

  \section{Numerical Methods} % (fold)
  \label{sec:numerical_methods}
    \begin{itemize}
      \item classification of partial differential equations
      \item finite difference methods
    \end{itemize}
  % section numerical_methods (end)

  \section{Binary Black Hole Evolution and Results} % (fold)
  \label{sec:binary_black_hole_evolution_and_results}
    \begin{itemize}
      \item show some videos
      \item gravitational lensing
      \item inspiral, merge and ringdown
    \end{itemize}
  % section binary_black_hole_evolution_and_results (end)

  \section{Conclusion and References} % (fold)
  \label{sec:references}

  % \href{run:videos/Movies - SXS - Simulating eXtreme Spacetimes.mp4}{Click for video}

  \nocite{*}
  \bibliographystyle{babplain}
  \bibliography{references}

  % section references (end)
\end{document}